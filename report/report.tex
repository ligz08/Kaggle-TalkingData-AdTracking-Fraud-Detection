% --------------------------------------------------------------
% This is all preamble stuff that you don't have to worry about.
% Head down to where it says "Start here"
% --------------------------------------------------------------
 
\documentclass[12pt]{article}
 
\usepackage[margin=1in]{geometry} 
\usepackage{amsmath,amsthm,amssymb}
\usepackage{listings}
\usepackage{bm}
\usepackage{hyperref}

 
\newcommand{\N}{\mathbb{N}}
\newcommand{\Z}{\mathbb{Z}}
 
\newenvironment{theorem}[2][Theorem]{\begin{trivlist}
\item[\hskip \labelsep {\bfseries #1}\hskip \labelsep {\bfseries #2.}]}{\end{trivlist}}
\newenvironment{lemma}[2][Lemma]{\begin{trivlist}
\item[\hskip \labelsep {\bfseries #1}\hskip \labelsep {\bfseries #2.}]}{\end{trivlist}}
\newenvironment{exercise}[2][Exercise]{\begin{trivlist}
\item[\hskip \labelsep {\bfseries #1}\hskip \labelsep {\bfseries #2.}]}{\end{trivlist}}
\newenvironment{reflection}[2][Reflection]{\begin{trivlist}
\item[\hskip \labelsep {\bfseries #1}\hskip \labelsep {\bfseries #2.}]}{\end{trivlist}}
\newenvironment{proposition}[2][Proposition]{\begin{trivlist}
\item[\hskip \labelsep {\bfseries #1}\hskip \labelsep {\bfseries #2.}]}{\end{trivlist}}
\newenvironment{corollary}[2][Corollary]{\begin{trivlist}
\item[\hskip \labelsep {\bfseries #1}\hskip \labelsep {\bfseries #2.}]}{\end{trivlist}}
 
\begin{document}
 
% --------------------------------------------------------------
%                         Start here
% --------------------------------------------------------------
 
%\renewcommand{\qedsymbol}{\filledbox}
 
\title{TalkingData AdTracking Fraud Detection Challenge}%replace X with the appropriate number
\author{Guozhen Li, Hairu Lang \\ %replace with your name
STA 208 - Statistical Machine Learning - Final Project} %if necessary, replace with your course title

\maketitle


\section{Problem Description}
This problem is a data analysis challenge from Kaggle (www.kaggle.com), originally found at  \url{https://www.kaggle.com/c/talkingdata-adtracking-fraud-detection}.
The data provider, TalkingData, tracks a large base of mobile device activity, 
and wants to understand which ad clicks end up with app downloads, 
and which ones lead to nothing (thus are suspects of fraud clicks).
We are provided with a training data set of over 180 million clicks, 
and the objective is to predict whether a click leads to an app download.

For each click, from the raw training data set we know about:
\begin{itemize}
	\item IP address
	\item app of marketing
	\item device type
	\item mobile device operating system type
	\item mobile ad publisher channel
	\item click time (date and time, down to seconds)
	\item whether a download occurs at the end
	\item download time, if a download occurs
\end{itemize}

This data analysis task poses, at least, the following challenges:
\begin{enumerate}
	\item Extract useful features from the seemingly dispersed information found in the raw training data.
	\item Handling of the extra-large data set (over 180 million samples, as an over 7.5GB text file), which can hardly fit into the memory of a normal personal computer.
	\item Efficient and effective learning method to learn from the data.
\end{enumerate}


\section{Data Exploration}

\section{Feature Engineering and Data Pre-processing}

\section{Modeling Predicting}

\section{Model Evaluation}

\section{Conclusions}

 
% --------------------------------------------------------------
%     You don't have to mess with anything below this line.
% --------------------------------------------------------------
 
\end{document}